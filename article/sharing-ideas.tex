%%
%% 研究報告用スイッチ
%% [techrep]
%%
%% 欧文表記無しのスイッチ(etitle,jkeyword,eabstract,ekeywordは任意)
%% [noauthor]
%%

%\documentclass[submit,techreq]{ipsj}
\documentclass[submit,techreq,noauthor]{ipsj}

\usepackage[dvips]{graphicx}
\usepackage{latexsym}

\def\Underline{\setbox0\hbox\bgroup\let\\\endUnderline}
\def\endUnderline{\vphantom{y}\egroup\smash{\underline{\box0}}\\}
\def\|{\verb|}

\newcommand{\TED}{\textsf{TED}}
\newcommand{\TEDx}{\TED${}^{\textsf{x}}$}
\newcommand{\TEDxTokyo}{\TEDx\textsf{Tokyo}}
\newcommand{\TEDxKyoto}{\TEDx\textsf{Kyoto}}

\setcounter{巻数}{0}%vol53=2012
\setcounter{号数}{0}
\setcounter{page}{1}

\begin{document}

\title{アイディアの共有手段としてのエンタテイメント}

%\etitle{How to Prepare Your Paper for IPSJ Journal \\
%(version 2012/10/12)}

\affiliate{OsakaU}{大阪大学\\
Osaka University, Suita, Osaka 565-0871, Japan}

\affiliate{KUFS}{京都外国語大学\\
Kyoto University of Foreign Studies, Kyoto 615-8558, Japan}

\author{金谷 一朗}{Ichiroh Kanaya}{OsakaU}[kanaya@pineapple.cc]
\author{ジェイ クラパーキ}{Jay Klaphake}{KUFS}

\begin{abstract}
建築家リチャード・ソウル・ワーマンは1976年に「情報アーキテクチャ」という言葉を作り,
情報における「アーキテクチャ」の必要性を訴えた.
彼が1984年に設立した非営利のカンファレンス \TED\ (Technology, Entertainment, Design) は
彼の情報アーキテクチャの概念を最もよく表した例である.
\TED カンファレンスは元々学術,エンタテイメント,デザイン分野の研究者がアイディアを共有する場であったが,
現在は分野を絞らず,またそのプレゼンテーションフォーマットを \TEDx として外部にライセンスすることで,
広く社会的影響を与えている.本報告では,\TED スタイルと呼ばれるワーマンの設計した情報共有のための
アーキテクチャを一例として,アイディアの共有手段としてのエンタテイメントの可能性を探る.
\end{abstract}

\begin{jkeyword}
555カンファレンス,\TED カンファレンス,アイディア共有,エンタテイメント,プレゼンテーション形式,
情報アーキテクチャ
\end{jkeyword}

\maketitle

\section{はじめに}

% 問題の定義

情報に構造を与え情報を理解しやすくする研究には長い歴史があり,
それは少なくとも紀元前335年にアリストテレスによって書かれた「詩学」\cite{a}にまで遡る.

一方で,サイエンスとテクノロジーは人類共有の財産であるが,その知見に構造を与え,
情報を共有することはは必ずしも簡単ではない.
サイエンス,テクノロジーは数学という高度な抽象度と正確性を持つ道具を駆使するため,
他の学問分野と比較しても一般に「敷居が高い」からである.

学問,とりわけサイエンスやテクノロジーに関する知見の共有は人類の持続的発展のために特に重要であり,
その試みは学会や教育機関などによって無数に行われている.
近年のエデュテイメント (edutainment) や ``Playful Learning'' は特に子供向け教育を念頭に置いた
活動であるが,それらは子供のみならず一般社会人や専門以外の研究者にもその十分親しまれている.\cite{nu}

% Passion Based Learning

% 王立協会 

本報告は,特に近年のサイエンス,テクノロジー分野の知見を一般に共有するための手段を,
その目的において成功しつつある \TED カンファレンスおよび \TEDx カンファレンスを内側から俯瞰することで,
共有しようとするものである.


\section{TEDカンファレンスの取り組み}

\subsection{カンファレンスのデザイン}

リチャード・ソール・ワーマンは当時非公開であった \TED カンファレンスを1984年に設立するにあたって,
以下のルールを設けた.
\begin{itemize}
\item 1回の講演のトピックはひとつだけに制限される.
\item ひとりの講演者の持ち時間は最大18分であり,一般にはより短い持ち時間のみが与えられる.
\item 質疑応答の時間は一部の例外を除いて用意されず,発表後の交流会が質疑応答の時間にあてられる.
\item 講演者は挨拶,自己紹介を行わず,講演のアウトラインを冒頭に話すことも無い.
\item スライドやビデオの使用は推奨される.
\item ポインタは使用されない.
\item ポディウムは推奨されない.
\end{itemize}
ワーマンは神経科学の知見から18分という長さを割り出したのではなく,
過去の著名な講演の殆どが18分以内であることを根拠に講演の最大長を決定したと考えられる.\cite{cg}
また2001年にワーマンの後を受けた \TED キュレータである
クリス・アンダーソンは以下のように述べている.\cite{cgweb}
\begin{quote}
It [18 minutes] is long enough to be serious and short enough to hold people's attention.
It turns out that this length also works incredibly well online. It's the length of a 
coffee break. So, you watch a great talk, and forward the link to two or three people. 
It can go viral, very easily. The 18-minute length also works much like the way Twitter 
forces people to be disciplined in what they write. By forcing speakers who are used to 
going on for 45 minutes to bring it down to 18, you get them to really think about what 
they want to say. What is the key point they want to communicate? It has a clarifying 
effect. It brings discipline.
\end{quote}

講演の質疑応答時間を削っているのは,\TED カンファレンスが発表者と聴講者との十分な交流に力を注いでいる
ためである.\TED カンファレンスではおよそ5日間のカンファレンスに対し参加者を1,000人以下に絞っており,
またセッション間の交流時間を毎回1時間以上とることで,発表者,聴講者および聴講者同士の交流時間を十分に
確保するようにしている.

挨拶,自己紹介の排除は,トピックをアイディアに集中させるためであり,
また講演そのものに余分なコンテクストを残さないための措置である.

ポインタおよびポディウムの非推奨はひとつの理由に基づく.ポディウムの存在は講演者と聴講者の間に文字通り壁を
作るものであり,ワーマンによれば講演においては避けるべきものである.
ポディウムが無い場合,映像を送り出す装置(PC)は手元に無いためポインタはレーザポインタを使って
光学的に重畳することになるが,これは複数スクリーンに対応せず,
また後に述べるビデオアーカイブにもポインタ軌跡を残すことが困難であるため推奨されない.

導入,関連研究,原理,実験方法,実験結果,考察,結言と言った理工系論文に見られる章立ては用いられない.
あえて理工系論文の章立てに当てはめると,\TED で行われる講演は導入部のみである.

\subsection{ビデオアーカイブ}

\TED カンファレンスは2006年から講演のビデオアーカイブをクリエイティブ・コモンズ (CC-BY-NC-ND) ライセンス
下で無償公開している.(一部の動画はダブルライセンスによる公開を行っている.)

ビデオアーカイブ公開の目的はアイディアの共有であるが,それが \TED ビデオであることを示す様々な工夫が
盛り込まれている.

講演者は丸い赤絨毯の上に立つ.赤絨毯のサイズは直径2.5[m]から5[m]程度まで毎回異なる.
赤絨毯は講演者の移動範囲を制限し,それによって照明およびカメラの照準を決めやすくしている.

カメラは前方4台,後方1台の合計5台を用いることが多く,前方の1台はクレーン,レールを用いることが多い.
これは一般的な講演の撮影と,音楽コンサートの撮影の中間程度の設備である.

また \TED ビデオの特徴として,客席に青い照明をあてることがあげられる.
講演者の赤絨毯とのカラーのコントラストもあるが,青い照明で客席の照度を上げつつ,
視覚的には暗いという印象を与えるためである.
客席には青照明によって十分な照度が与えられるため,
ビデオアーカイブを見ると聴講者の表情を読み取ることが出来,
聴講者もまたカンファレンスの一部であることがビデオを通して伝わる.

講演者が18分を超える講演を行った場合でも,ビデオアーカイブは18分に制限される.


% Non commercial

\subsection{TEDx}

\TED カンファレンスが2009年にパトリック・ニューウェル (Patrick Newel) とトッド・
ポーター (Tod Porter) に independently organized TED event と
しての \TEDxTokyo ライセンスを与えたのが \TEDx イベントの始まりである.

\TEDx ライセンス保持者は \TED カンファレンスが提示する \TEDx ライセンスに従わねばならない. \TEDx ライセンスは1年間
有効で,授与,更新,譲渡の可否の判断は \TED カンファレンスに委ねられる.





\section{おわりに}

...

\begin{acknowledgment}
...
\end{acknowledgment}



\begin{thebibliography}{10}

\bibitem{a}
アリストテレース: 詩学; 岩波書店, 1997.

\bibitem{nu}
上田信行, 中原淳: プレイフル・ラーニング; 三省堂, 2012.

\bibitem{rsw}
Richard Saul Wurman: \textit{Information Architects;}
Graphis Inc., 1997.

\bibitem{cg}
Camine Gallo: \textit{Talk like TED: The 9 Public-Speaking Secrets of the World's Top Minds;}
St. Martin's Press, 2014.

\bibitem{cgweb}
Camine Gallo: \textit{The Science Behind TED's 18-Minute Rule;}
\urle{http://www.linkedin.com/today/post/article/20140313205730-5711504-the-science-behind-ted-s-18-minute-rule}
(2014-07-21).

\bibitem{historyofted}
TED: \textit{History of TED (online);}
\urle{https://www.ted.com/about/our-organization/history-of-ted} (2014-07-21).

\end{thebibliography}




%% 以下は無視されます

\begin{biography}
\profile{m}{情報 太郎}{1970年生.1992年情報処理大学理学部情報科学科卒.
1994年同大大学院修士課程了.同年情報処理学会入社.オンライン出版の研究
に従事.電子情報通信学会,IEEE,ACM 各会員}
%
\profile{n}{処理 花子}{1960年生.1982年情報処理大学理学部情報科学科卒.
1984年同大大学院修士課程了.1987年同博士課程了.理学博士.1987年情報処
理大学助手.1992年架空大学助教授.1997年同大教授.オンライン出版の研究
に従事.2010年情報処理記念賞受賞.電子情報通信学会,IEEE,IEEE-CS,ACM
各会員}
%
\profile{s}{学会 次郎}{1950年生.1974年架空大学大学院修士課程了.
1987年同博士課程了.工学博士.1977年架空大学助手.1992年情報処理大学助
教授.1987年同大教授.2000年から情報処理学会顧問.オンライン出版の研究
に従事.2010年情報処理記念賞受賞.情報処理学会理事.電子情報通信学会,
IEEE,IEEE-CS,ACM 各会員}
%
\end{biography}



\end{document}
